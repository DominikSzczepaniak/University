\documentclass[12pt]{article}
\usepackage{amsmath}
\usepackage[T1]{fontenc}
\usepackage{graphicx}

\title{MDL 4 30.10}
\author{Dominik Szczepaniak}
\begin{document}

\maketitle
Zrobione:
\begin{tabular}{|| c c c c c c c c c c c||}
    \hline
    1D & 2 & 3 & 4 & 5 & 6 & 7 & 8 & 13  \\
    \hline
    Y & Y & Y & Y & Y & Y & Y & Y & Y
\end{tabular}

\bgroup\obeylines

\section{Zadanie 1}
Rekurencyjne rozwiązanie:
Załóżmy, że n okręgów dzieli płaszczyzne na $r_n$ regionów. Dodajemy jeden okrąg (on przecina każdy pozostały okrąg w dwóch miejscach), więc mamy 2n przecięć. Jeśli p0 i p1 są kolejnymi przecięciami to łuk z p0 do p1 dzieli jeden z $r_n$ regionów na dwa. Zrobimy to 2n razy i mamy:
$r_{n+1} = r_n + 2n$


Rozważmy przykład idealny: nowy okrąg n przecina n-1 pozostałych okręgów w 2 punktach oraz nie ma takiego punktu gdzie trzy okręgi się spotykają. Załóżmy graf G(V, E) gdzie V jest zbiorem punktów przecięć okręgów a E jest zbiorem łuków które łączą dwa punkty V. 
Każdy okrąg ma 2(n-1) punktów przecięć, mamy n okręgów i każdy punkt przecięcia należy do 2 okręgów. W takim razie ilość tych punktów równa jest
$V = \frac{1}{2}*(2(n-1))*n = n*(n-1)$
Każdy wierzchołek jest stopnia 4. Z zadania o uscisnieciach rąk(każda krawędź musi łączyć dwa wierzchołki, więc każda krawędź dodaje 2 do wyniku sumy stopnii wszystkich wierzchołków, wiec $\sum_{v \in V}{} deg(v) = 2E => E = \frac{1}{2} \sum{v \in V}{} deg(v)$) mamy:
$E = \frac{1}{2} * \sum_{v \in V}{} deg(v) = \frac{1}{2} * \sum_{v \in V}{} 4 = \frac{1}{2} * 4V = 2V = 2n*(n-1)$
Z twierdzenia Eulera o grafach planarnych:
V - E + F = 2
Gdzie F to ilosc regionow uformowanych przez graf planarny. Wtedy
$F = 2 + E - V = 2 + 2n*(n-1) - n(n-1) = 2 + 2n^2 - 2n - n^2 + n = n^2 - n + 2$

Dowód eulera o grafach planarnych:
Niech $G = (V,E)$ będzie grafem, użyjemy indukcji na krawędziach. Dla E = 0 mamy 1 wierzchołek i mamy jedno podzielenie płaszczyzny, więc 1-0+1 = 2 czyli zachodzi. Teraz załóżmy dla kroku indukcyjnego, że |E| = n no i V - E + F = 2. Rozwazmy graf gdzie |E| = n+1.

1. Drzewo. Wtedy V = E+1 czyli V = n+2. Drzewa nie maja podziału płaszczyzny (bo nie ma cykli), więc F = 1.
Takze V - E + F = 2 => n+2 - n-1 + 1 = 2-1+1 = 2 czyli prawda.

2. Graf z cyklami. Niech p będzie krawędzią z cyklem. Wtedy $G' = (V, E/p)$ (graf z usunieta krawedzia p). Wtedy mamy E' = E - 1. Skoro p było w cyklu to dzieliło płaszczyzne, więc usunięcie tego usuwa jedno podzielenie płaszczyzny (F' = F - 1). Skoro G ma n+1 krawędzi, to G' ma n i z indukcyjnej hipotezy V' - E' + F' = 2. Po podstawieniu E' = E-1, V' = V, F' = F-1 mamy 
$V - E+1 + F-1 = 2 => V - E + F = 2$ 
Robiąc to indukcyjnie wiele razy (aż dojdziemy do grafu bez cykli powyższe wyrażenie jest prawdziwe bo otrzymamy drzewo).
Czyli twierdzenie zachodzi, także V-E+F = 2.


\section{Zadanie 2}
Na każdy schodek możemy wejść albo z poprzedniego albo z przedpoprzedniego, więc mamy $F_n = F_{n-1} + F_{n-2}$.
Także dla n stopni mamy n-tą liczbę fibonacciego.
\section{Zadanie 3}
Jeśli wyjmiemy jedno pole koloru białego i czarnego to mamy 62 pola - 31 białych i 31 czarnych, więc układadamy domina tak, aby leżało na jednym czarnym i jednym białym polu.
\section{Zadanie 4}
Niech będą 3 wiersze.
Mamy dwa kolory więc z zasady szufladkowej w każdym wierszu będą 2 kolory takie same.  
3 pola możemy pokolorować na $\binom{3}{2}$ = 3 sposobów. 
Jeśli odwrócimy kolory to dla 6 kolumn może być każda kolumna innego koloru.
Ale jeśli dodamy jeszcze jedną kolumnę no to muszą być przynajmniej dwie kolumny tego samego koloru, stąd dla 3x7 zachodzi i dla żadnej z mniejszych nie zajdzie.

Dla 3x6 kontprzykład:
.XXX..
X.X.X.
XX...X
\section{Zadanie 5}
Dla 3x6 nie zachodziło 
Dla dwóch wierszy nie będzie odpowiedzi, bo potrzeba nam przynajmniej 3 wierszy aby stworzyć prostokąt. Rozpatrzyliśmy już 3 wiersze, więc spójrzmy na 4.
Żeby było mniej pól łącznie to liczba kolumn musi być równa 4 lub 5 (dla 6 jest więcej pól, a dla 3 nie ma rozwiązania)
W takim razie 4x4:
kontprzykład
CZCZ 
CCZZ 
ZCZC
ZZCC
4x5:
kontprzykład
CZCZ 
CCZZ 
ZCZC
ZZCC
CZZC 
Dla 5 kolumn możemy wybrać tylko 3 i 4, ale to wiemy z poprzedniego zadania ze nie zachodzi. Także nie ma co dalej patrzeć, bo dla 6 mamy 3, a to jest to samo co 3x6.
Także 3x6 to najmniejsza ilość pól która da nam prostokąt. 


\section{Zadanie 6}
Możemy rozłożyc ludzi albo na przemian: wtedy jakiś chłopak ma wokół siebie dwie dziewczyny - sprzeczność.
Możemy rozłożyć ludzi w dwa bloki - wtedy jakaś dziewczyna będzie miała wokół siebie dwie dziewczyny. 
Aby udało się rozłożyć tak jak chcemy potrzebowalibyśmy ciągu CCDCC... czyli na każdą dziewczynę przypada dwóch chłopaków. Tutaj niestety nie możemy tego dostać bo jest ich po równo.

Zasada szufladkowa na 13 siedzeń?

Załóżmy, że nie będzie takiej osoby. Wtedy po obu stronach będą chłopacy. Najoptymalniejszy sposób to, żeby tą osobą była dziewczyna. Wtedy pozostają 23 wolne miejsca i 12 dziewczyn oraz 11 chłopaków. Oczywiście nie możemy układać ich na przemian, bo wtedy jakiś chłopak będzie miał dookoła siebie dwie dziewcyzny. W takim razie musimy zrobić dwa duże bloki. No ale to też jest sprzeczne, bo wtedy jakaś dziewczyna będzie miała dookoła siebie dziewczyny. 
\section{Zadanie 7}
Wsród n+1 muszą być przynajmniej dwie kolejne liczby, a te są względnie pierwsze.
Weźmy dwie liczby kolejne k i k+1 
Załóżmy, że nie są względnie pierwsze 
Wtedy nwd(k, k+1) = x, x>1
Wtedy x dzieli k i x dzieli k+1 
Czyli takze x dzieli (k+1) - k = 1
No ale skoro x dzieli jeden to x = 1, co jest sprzeczne
Wiec k i k+1 musza byc wzglednie pierwsze. 

\section{Zadanie 8}

Reszty z dzielenia z n:
będzie istnieć albo
2 szufladki po 2 przedmioty 
1 szufladka z 3 przedmiotami 
jeśli ich różnica jest podzielna przez 2n to mamy to co chcieliśmy, jeśli nie jest to w takim razie dodanie n do tej różnicy da nam liczbę podzielną przez 2n.
Są one postaci k = z mod n 
czyli postaci k = z + cn mod 2n
Gdzie c = {0, 1}
Rozpatrzmy przypadki:
k1 = z + c1 * n mod 2n
k2 = z + c2 * n mod 2n 

1. c1,c2 = 0, 1
Wtedy różnica przystaje n mod 2n 
No ale możemy dobrać taką resztę z dzielenia j, że z + j + n = 0 mod 2n (wiemy, że istnieje, bo była w jakiejś szufladce)
TO WYKAZAĆ TROCHE LEPIEJ
2. c1, c2 = 0, 0
Wtedy oczywiście róznica przystaje do 0 mod 2n
3. c1, c2 = 1, 1
Wtedy też różnica przystaje do 0 mod 2n
4. c1, c2 = 1, 0
Analogicznie do 1. 


\section{Zadanie 13}
$9^{8^{7^{6^{5^{4^{3^{2^{1}}}}}}}}$
Wypiszmy kolejne reszty:
9
81
29
61
49
41
69
21
89
01
09 <- tu się zapętlamy, więc cykl ma 10 wyrazów

Także musimy się dowiedzieć jaką reszte z dzielenia przez 10 ma $8^{7^{6^{5^{4^{3^{2^{1}}}}}}}$
Wypiszmy teraz reszty 8:
8
4
2
6
8 <- pętla 
Cykl długości 4

Czyli reszta z dzielenia przez 4 liczby $7^{6^{5^{4^{3^{2^{1}}}}}}$
Reszty 7 z dzielenia przez 4:
3
1
3 <- cylk długości 2
Czyli reszta z dzielenia przez 2 liczby $6^{5^{4^{3^{2^{1}}}}}$
6 ma tylko jedną resztę - 6, w takim razie reszta równa się 0.
Także z siódemek wychodzi 3.
Z ósemek 3 liczba to 2
Wtedy z 9 jest 81.
Wynik więc to 81


\egroup
\end{document}