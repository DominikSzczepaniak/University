\documentclass[12pt]{article}
\usepackage{amsmath}
\usepackage[T1]{fontenc}
\usepackage{graphicx}

\title{MDL 26.10 Zadanie domowe}
\author{Dominik Szczepaniak}
\begin{document}

\maketitle
\bgroup\obeylines
\section{Zadanie 7}
Wybieramy n+1 liczb spośród kolejnych 2n liczb naturalnych.
Zauważmy, że każda liczba jest zapisana jako albo $2^k$ * liczba nieparzysta. Liczby nieparzyste są ze zbioru {1,3,5,...,2n-1}. Więc co najwyżej możemy wziąc n takich wartości. 
Teraz włóżmy nasze n+1 liczb do odpowiednich szufladek która mówi o największym nieparzystym dzielniku. Zauważmy, że jeśli wzięliśmy n+1 liczb to musi istnieć szufladka, która ma co najmniej 2 elementy.
Wejdźmy więc do tej szufladki. Obie te liczby są postaci $2^k * l$, gdzie l to ta największa liczba nieparzysta, która dzieli tą liczbę.
Pierwszą liczbą niech będzie $2^k * l$, a drugą $2^m * l$ i niech k>m (równie nie mogą być, bo każda liczba ma być różna).
Oznaczmy te liczby jako a i b.
Więc 
a = $2^k * l$, 
b = $2^m * l$ 
$\frac{a}{b} = \frac{2^k * l}{2^m * l}$ => $a = 2^{k-m} * b$
Także b dzieli a. 

\egroup
\end{document}