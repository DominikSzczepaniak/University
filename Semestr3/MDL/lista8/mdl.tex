\documentclass[12pt]{article}
\usepackage{amsmath}
\usepackage[T1]{fontenc}
\usepackage{graphicx}
\usepackage{amsfonts}
\usepackage{listings}
\newcommand{\floor}[1]{\left\lfloor #1 \right\rfloor}	% podłoga
\newcommand{\ceil}[1]{\left\lceil #1 \right\rceil}		% sufit
\newcommand{\fractional}[1]{\left\{ #1 \right\}}		% część ułamkowa {x}
\newcommand{\abs}[1]{\left| #1 \right|}					% wartosc bezwzgledna / moc
\newcommand{\set}[1]{\left \{ #1 \right \}}				% zbiór elementów {a,b,c}
\newcommand{\pair}[1]{\left( #1 \right)}				% para elementów (a,b)
\newcommand{\Mod}[1]{\ \mathrm{mod\ #1}}				% lekko zmodyfikowane modulo
\newcommand{\comp}[1]{\overline{ #1 }} 					% dopełnienie zbioru 
\newcommand{\annihilator}{\mathbf{E}}					% operator E
\newcommand{\seqAnnihilator}[1]{\annihilator \left\langle #1 \right\rangle} % E(a_n)
\newcommand{\sequence}[1]{\left\langle #1 \right\rangle} % <a_n>
\title{MDL 8 30.11}
\author{Dominik Szczepaniak}
\begin{document}

\maketitle
Zrobione:
\begin{tabular}{|| c c c c c c c c c c c||}
    \hline
    1 & 2 & 4 & 5 & 7 & 8 & 9   \\
    \hline
    Y & N & Y & Y & N & Y & Y 
\end{tabular}

\bgroup\obeylines
\section{Zadanie 1}
a) Z wykładu $\prod_{i=1}^{\infty} {\frac{1}{1-x^i}}$
b) $\prod_{i=1}^{\infty} {{1+x^{2i-1}}}$
c) $\prod_{i=1}^{m-1} {\frac{1}{1-x^i}}$
d) $\prod_{i=1}^{\infty} {{1+x^{2^i}}}$
\section{Zadanie 2}
SKIP
\section{Zadanie 4}
Zakładam, że identyczny to znaczy, że wszystkie krawędzie są takie same w tym grafie.
Zakładam też ze graf jest podany jako lista sąsiedztwa
\begin{lstlisting}
def is_identical(G, H):
    if(len(G) != len(H)):
        return False
    visited = [False] * len(G)
    for v in range(1, len(G+1)):
        n=0
        for adj in G[v]:
            visited[adj] = True 
            n+=1
        for adj in H[v]:
            if(!visited[adj]):
                return False
            visited[adj] = False 
            n-=1
        if(n!=0):
            return False
    return True
\end{lstlisting}
Przechodzimy po każdym wierzchołku i jego krawędziach w grafie G a później w grafie H, wiec amortyzuje się to do n+m (bo jest 2n wierzchołków i łączna liczba krawędzi nie przekracza m).

\section{Zadanie 5}
a) 
W macierzowej to jest po prostu suma na kolumnie dla tego wierzchołka, w listowej jest to długość listy sąsiedztwa.
$O(n) vs O(1)$
b)
Wszystkie krawędzie grafu to dla macierzowej po prostu przejście po całej macierzy, czyli $n^2$.
Dla listowej jest to przejście po wszystkich krawędziach dwa razy (bo jeśli w 2 była 1, to w 1 będzie 2, czyli 2x).
$O(n^2) vs O(m)$
c)
W macierzowej po prostu odniesienie się do tablicy w tym punkcie
W listowej trzeba w najgorszej opcji przejść po wszystkich sąsiadach u, których może być v.
$O(1) vs O(n)$
d)
W macierzowej przestawienie wartości z 1 na 0 w odpowiednim miejscu tablicy.
W listowej jest to w najgorszej opcji przejście przez wszystkich sąsiadów i usunięcie v.
$O(1) vs O(n)$
e) 
W macierzowej jest to przestawienie z 0 na 1.
W listowej jest to pushback dla u.
$O(1) vs O(1)$

\section{Zadanie 7}
Weźmy $G = {V, E} = {{1 2}, {}}$
Wtedy
$G_1 = {V, E} = {{2}, {}}$
$G_2 = {V, E} = {{1}, {}}$
$G_1$ i $G_2$ są spójne, a G nie jest, czyli teza nie zachodzi.





\section{Zadanie 8}
Załóżmy nie wprost, że istnieją dwie równe najdłuższe scieżki i nie przechodzą przez ten sam wierzchołek. Niech jedna droga będzie z (v1, u1) a druga (v2, u2). Weźmy wtedy drogę z v1 do v2 i przedłużmy tą drogę do u2. Skoro (v2, u2) jest nadłuższą trasą która nie ma żadnego wspólnego wierzchołka z (v1, u1), a graf jest spójny do możemy przedłużyć tą trasę idąc z v1 do v2 a później znowu tą samą trasą. 
Mamy sprzeczność, bo ta trasa jest dłuższa, więc dwie najdłuższe ścieżki mają wspólny wierzchołek.

\section{Zadanie 9}
Weźmy dowolny graf G. Załóżmy, że nie jest spójny. Weźmy dowolne dwa wierzchołki i nazwijmy je v i u. Załóżmy, że te wierzchołki nie mają między sobą krawędź w G. W takim razie w G' istnieje krawędź między nimi. 
Załóżmy teraz, że te wierzchołki mają między sobą krawędź w G. W takim razie są w tym samym komponencie w grafie G. Skoro G nie jest spójny to możemy znaleźć taki wierzchołek w, który nie należy do tego komponentu (czyli nie ma drogi między nimi, czyli nie ma krawędzi między nimi). Wtedy w G' będzie istniec zarówno krawędź vw jak i uw, czyli droga z v to u istnieje przez krawędź w. Możemy zastosować powyższe myślenie dla każdej pary wierzchołków, przez co wszystkie wierzchołki będą miały drogę do siebie, co dowodzi, że G' jest wtedy spójny. 






\egroup
\end{document}