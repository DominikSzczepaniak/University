\documentclass[12pt]{article}
\usepackage{amsmath}
\usepackage[T1]{fontenc}
\usepackage{graphicx}
\usepackage{amsfonts}
\usepackage{listings}
\newcommand{\floor}[1]{\left\lfloor #1 \right\rfloor}	% podłoga
\newcommand{\ceil}[1]{\left\lceil #1 \right\rceil}		% sufit
\newcommand{\fractional}[1]{\left\{ #1 \right\}}		% część ułamkowa {x}
\newcommand{\abs}[1]{\left| #1 \right|}					% wartosc bezwzgledna / moc
\newcommand{\set}[1]{\left \{ #1 \right \}}				% zbiór elementów {a,b,c}
\newcommand{\pair}[1]{\left( #1 \right)}				% para elementów (a,b)
\newcommand{\Mod}[1]{\ \mathrm{mod\ #1}}				% lekko zmodyfikowane modulo
\newcommand{\comp}[1]{\overline{ #1 }} 					% dopełnienie zbioru 
\newcommand{\annihilator}{\mathbf{E}}					% operator E
\newcommand{\seqAnnihilator}[1]{\annihilator \left\langle #1 \right\rangle} % E(a_n)
\newcommand{\sequence}[1]{\left\langle #1 \right\rangle} % <a_n>
\title{Zadanie domowe lista 8}
\author{Dominik Szczepaniak}
\begin{document}
\maketitle

\section{Zadanie 4}
Zakładam, że identyczny to znaczy, że wszystkie krawędzie są takie same w tym grafie.
Zakładam też ze graf jest podany jako lista sąsiedztwa
\begin{lstlisting}
def is_identical(G, H):
    if(len(G) != len(H)):
        return False
    visited = [False] * len(G)
    for v in range(1, len(G+1)):
        n=0
        for adj in G[v]:
            visited[adj] = True 
            n+=1
        for adj in H[v]:
            if(!visited[adj]):
                return False
            visited[adj] = False 
            n-=1
        if(n!=0):
            return False
    return True
\end{lstlisting}
Przechodzimy po każdym wierzchołku i jego krawędziach w grafie G a później w grafie H, wiec amortyzuje się to do n+m (bo jest 2n wierzchołków i łączna liczba krawędzi nie przekracza m).

\end{document}