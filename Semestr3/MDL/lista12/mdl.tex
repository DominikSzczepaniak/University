\documentclass[12pt]{article}
\usepackage{amsmath}
\usepackage[T1]{fontenc}
\usepackage{graphicx}
\usepackage{amsfonts}
\usepackage{listings}
\newcommand{\floor}[1]{\left\lfloor #1 \right\rfloor}	% podłoga
\newcommand{\ceil}[1]{\left\lceil #1 \right\rceil}		% sufit
\newcommand{\fractional}[1]{\left\{ #1 \right\}}		% część ułamkowa {x}
\newcommand{\abs}[1]{\left| #1 \right|}					% wartosc bezwzgledna / moc
\newcommand{\set}[1]{\left \{ #1 \right \}}				% zbiór elementów {a,b,c}
\newcommand{\pair}[1]{\left( #1 \right)}				% para elementów (a,b)
\newcommand{\Mod}[1]{\ \mathrm{mod\ #1}}				% lekko zmodyfikowane modulo
\newcommand{\comp}[1]{\overline{ #1 }} 					% dopełnienie zbioru 
\newcommand{\annihilator}{\mathbf{E}}					% operator E
\newcommand{\seqAnnihilator}[1]{\annihilator \left\langle #1 \right\rangle} % E(a_n)
\newcommand{\sequence}[1]{\left\langle #1 \right\rangle} % <a_n>
\title{MDL 12 11.01.2024}
\author{Dominik Szczepaniak}
\begin{document}

\maketitle

\bgroup\obeylines
\section{Zadanie 1} %done
Niech każdy kwadrat będzie innego koloru (pokolorujmy go dwoma kolorami). Wtedy możemy tylko iść z jednego koloru na drugi. Zauważmy, że mamy 14 kwadratów jednego koloru i 13 kwadratów drugiego koloru, z tym, że 14 kwadratów jest koloru tego co róg kostki, a środek kostki jest koloru innego niż ten co w rogu. W takim razie jeśli chcielibyśmy skończyć na środku to nie odwiedzimy 1 kwadratu.
\section{Zadanie 2} %skip
Podstawa indukcji 2:
1->2->3->4->1  (kwadratowo)
Cykl hamiltona istnieje.
Z założenia w Qn istnieje cykl hamiltona, zatem istnieją dwa wierzchołki które mają wspólną krawędź, niech to będą v i w.
Z konstrukcji Qn+1 wymiarowej kostki istnieją dwa Qn i Qn, t. że v ma krawędź z v' i w ma krawędź z w'. Skoro w Qn istniał cykl to istnieje ścieżka z v do w. Krawędź z w do w' nie znajduje się w tej ścieżce, bo nie była w Qn. Skoro w Qn' był cykl to istnieje ścieżka Hamiltona z w do w'. Z zał. konstrukcyjnego istnieje ściezka z v' do v i nie znajdowała się w Qn ani w Qn'. Jest ona inna niż krawędź od w do w'. Zatem istnieje ścieżka w tej kostce Qn+1 na mocy indukcji. 
\section{Zadanie 3} %done
Stwórzmy graf w którym jeśli uczniowie się przyjaźnią to mają krawędź do siebie.
Twierdzenie Diraca - jeśli graf jest grafem prostym (nasz jest) o co najmniej trzech wierzchołkach i minimalnym stopniu >= n (nasz jest) to graf zawiera cykl hamiltona.
W takim razie jeśli mamy cykl hamiltona <v1, v2, v3, ... vn> stwórzmy grupy biorać każdą dwójkę uczniów koło siebie w cyklu hamiltona. Drugim sposobem moze byc przesuniecie o 1 wyboru.

Musimy rozpatrzeć przypadki gdy mamy n=1 (czyli dwóch uczniów). Skoro mamy dwóch uczniów i każdy musi mieć jednego przyjaciela to przyjaźnią sie wzajemnie.

\section{Zadanie 5}%done
Stwórzmy nowy graf H, taki że, tworzymy nowy wierzchołek w, który jest połączony z każdym wierzchołkiem w G. 
Teraz zauważmy, że mamy tak:
$deg_G(v) + deg_G(u) + 2 = deg_H(v) + deg_H(u) >= n-1 + 2 >= n+1$
Z twierdzenia Ore'a wiemy więc, że graf H zawiera cykl hamiltona.
W takim razie jeśli nasz graf zawiera cykl hamiltona to możemy zacząć ten cykl hamiltona w dowolnym wierzchołku. Zacznijmy więc ten cykl w wierzchołku w. Czyli mamy cykl wierzchołków typu:
$<w, v_1, v_2, ..., v_{n-1}, v_n, w>$
W takim razie jeśli usuniemy wierzchołek w, to dostaniemy:
$<v_1, v_2, ..., v_{n-1}, v_n$> czyli ścieżke hamiltona. Czyli w grafie G istnieje ścieżka hamiltona.


\section{Zadanie 6} %done
Turniej to graf skierowany w ktorym kazda para wierzcholkow jest polaczona krawedzia a do b lub b do a. 
Indukcja po ilości wierzchołków (n):
n=1:
Tak, jest pojedycznym wierzchołkiem więc jest królem.

Załóżmy, że dla n zachodzi. Dla n+1:
Jeśli dla turnieju o wielkości n istniał król, to nowy wierzchołek to mamy dwie możliwości:
1. Król ma krawędź do nowego wierzchołka - wtedy dalej jest królem.
2. Nowy wierzchołek ma połączenie do króla. 
Teraz mamy dwa podprzypadki:
a) Jeśli do tego wierzchołka wchodzi jakakolwiek inna krawędź z wierzchołka do którego król miał bezpośrednie połączenie to król dalej ma ścieżke o dl 2 do tego wierzchołka. 
b) Jeśli do tego wierzchołka nie wchodzi żadna krawędź do której król miał bezpośrednie połączenie (dł 1) to ten wierzchołek ma krawędź do każdego z tych wierzchołków. No ale skoro ten wierzchołek ma takie same bezpośrednie połączenia co poprzedni król to może przejąć jego role, jeśli usuniemy króla, a jeśli ma dodatkowo połączenie do starego króla, to czyni go nowym królem, czyli w tym przypadku on będzie królem.

\section{Zadanie 7} %trudne
1. Pokażmy że nie istnieje dwóch króli 
2. Pokażmy, że przy naszym warunkach nie może istnieć jeden król. 


2. Indukcyjnie 
n=3:
Możemy mieć tylko graf a->b->c->a, a w nim znajduje się trzech króli, bo każdy wierzchołek jest królem.
Załóżmy, że w turnieju nie istnieje jeden król, ale wiemy, że musi istnieć jakiś król. Z punktu 1, wiemy, że nie może być dwóch króli, w takim razie liczba króli >=3. 
Oznaczmy tych króli przez po kolei $\set{a_1, a_2, ..., a_m}$.
Mamy dwa przypadki:
a) nasz nowy wierzchołek staje się królem 
Wtedy może być taki graf, gdzie każdy ze starych królów dojdzie do jakiegoś wierzchołka w dwóch krokach, a nowy wierzchołek nie będzie miał bezpośredniego połączenia do tego wierzchołka oraz będzie miał tylko wierzchołki wchodzące do obecnych króli. Wtedy nie dojdzie do tego 
b) wszyscy poprzedni królowie oprócz jednego tracą status króla

\section{Zadanie 8} %done
Algorytm będzie działał w sposób optymalny jeśli zminimalizuje liczbe chromatyczną grafu. 
W takim razie wiemy, że istnieje jakieś kolorowanie które da minimalną liczbę chromatyczną grafu, czyli jakiś wierzchołek będzie jakiegoś koloru. W takim razie jeśli algorytm wybierze ten wierzchołek jako pierwszy i później będzie szedł dokładnie tak samo jak optymalne kolorowanie to pokoloruje optymalnie i zminimalizuje liczbę chromatyczną grafu.
\section{Zadanie 9} %done
Indukcja dla drugiego punktu. Niech k = liczbie chromatycznej grafu G 
Dla k=1 musimy mieć (k * (k-1))/2 = (1 * 0) / 2 = 0 krawędzi, co jest prawdą.
Krok indukcyjny. Załóżmy, że mamy graf pokolorowany k kolorami i jest k(k-1)/2 krawędzi. 
Chcemy pokazać, że jak dodamy kolor, to musi być przynajmniej (k+1)*k / 2 krawędzi. 
Różnica wynosi $k^2/2+k/2 - k^2/2 + k/2 = k$ krawędzi. 
Czyli musimy pokazać, że jeśli dochodzi nowy kolor to musi dojść przynajmniej k krawędzi. To jest dość oczywiste, bo po prostu do każdego koloru musimy podłączyć nową krawędź, bo jeśli nie to można wybrać ten kolor który nie został podłączony i wtedy mamy poprawne kolorowanie grafu. 
4 - 4*3 / 2 = 6 krawędzi 


Mając jakieś kolorowanie musi być krawędź między każdymi dwoma zbiorami niezależnymi kolorów, inaczej pokolorowalibyśmy je na ten sam kolor.  
Mamy k zbiorów niezależnych kolorów. Jeśli między zbiorami niezależnymi nie ma krawędzi to możnaby było je pokolorować na ten sam kolor, czyli musi być krawędź między każdymi dwoma zbiorami niezależnymi, czyli k*(k-1). Ponieważ jeśli mamy dwa zbiory niezależne a i b, to jeśli jest krawędź z a->b to jest też z b->a, czyli musimy jeszcze podzielić na 2, stąd k*(k-1) / 2.

Jeśli to zachodzi dla algorytmu sekwencyjnego, to w szczegolnosci dla X(G), bo algorytm sekwencyjny moze wybrac dowolne k >= X(G).

\section{Zadanie 10} %done
Jeśli graf G nie jest spójny i mamy $G = G_1 \cup G_2 \cup ... \cup G_k$ to $X(G) = max(G_1, G_2, ..., G_k)$
Także zajmijmy się tylko grafami spójnymi. 

Z twierdzenie Brooks'a mamy, że dla każdego spójnego grafu nieskierowanego z maksymalnym stopniem k zachodzi, że X(G) jest co najwyżej k. Jeśli G jest kliką lub ma cykl długości nieparzystej to ma stopień k+1.
Sprawdzanie czy jest kliką jest w czasie wielomianowym $n^2$ (dla każdego wierzchołka sprawdzamy czy jest krawędź do każdego innego).
Sprawdzanie czy jest cykl o długości nieparzystej to algorytm BFS który zapisuje odwiedzane już wierzchołki i sprawdza długość trasy, czyli czas liniowy.


1. Wybierz dowolny wierzchołek v w grafie.
2. Rozpocznij od v i przemierzaj graf, odwiedzając każdy wierzchołek tylko raz.
3. Jeśli natrafisz na wierzchołek, który już odwiedziłeś, to oznacza, że znalazłeś cykl, teraz wystarczy sprawdzić jakiej długości on jest (dfs).
4. Jeśli odwiedzisz wszystkie wierzchołki w grafie, a nie znajdziesz cyklu, to oznacza, że graf nie posiada cyklu długości nieparzystej.

Ten algorytm działa, ponieważ jeśli graf posiada cykl długości nieparzystej, to koniecznie musi istnieć cykl, który zaczyna się i kończy w tym samym wierzchołku. Algorytm sprawdza wszystkie ścieżki w grafie, aż znajdzie taki cykl.

\section{Zadanie 11} %done
Niech graf dwudzielny będzie postaci A $\cup$ B. 
Niech do A trafiają wierzchołki o numerze nieparzystym, a do B wierzchołki o numerze parzystym.

Tworzymy graf w taki sposob, ze:
1 ma polaczenie z kazdym oprocz 2
3 ma polaczenie z kazdym oprocz 4
...
k ma polaczenie z kazdym oprocz k+1 
Wybieramy po kolei:
1, 2, 3, ..., 2n
Czemu dziala?
1 i 2 dostana 1 bo start i nie maja polaczenia ze soba
3 i 4 dostana 2 bo maja polaczenie odpowiednio z 2 i 1 
5 i 6 dostana 3 bo maja polaczenia odpowiednio z <2, 4> i <1, 3>
7 i 8 dostana 4 bo maja polaczenia odpowiednio z <2, 4, 6> i <1, 3, 5> itd.
k i k+1 dostana k+1 / 2 

Ostatnie k+1 bedzie postaci 2n, wiec dostanie n kolor, czyli to co chcieliśmy.


\section{Zadanie 12} %done
Z wykładu wiemy, że X(G) >= w(G), gdzie w to największa klitka w grafie.
Mamy, że:
$X(G) >= \frac{n}{a(G)}$
i $X(G') >= w(G') = a(G)$, gdzie $a(G)$ oznacza rozmiar największego zbioru niezależnego (zbioru w którym żadne dwa wierzchołki nie mają między sobą krawędzi) w danym grafie, a $w(G)$ oznacza wielkość największej kliki. 
Więc $X(G) * X(G') >= \frac{n}{a(G)} * a(G) = n$

Pokazać musimy, że 
1) $X(G) >= \frac{n}{a(G)}$ 
Zauważmy, że graf składa się z X(G) zbiorów niezależnych, a moc zbiorów niezależnych sumuje się do n.
W takim razie mamy:
$n = \sum_{i=1}^{X(G)} |C_i|$, gdzie $C_i$ oznacza i-ty zbiór niezależny.
No ale wiemy, że $|C_i| <= a(G)$, bo a(G) to największy zbiór niezależny.
Mamy więc:
$n = \sum_{i=1}^{X(G)} |C_i| <= \sum_{i=0}^{X(G)} a(G) = X(G)*a(G)$
Czyli $X(G) >= \frac{n}{a(G)}$
2) $w(G') = a(G)$ <- to jest z definicji obu. Jeśli a jest zbiorem w którym żadne dwa wierzchołki nie są połączone, a w jest zbiorem w którym każde dwa wierzchołki są połączone, to jeśli weźmiemy dopełnienie to mamy to samo.
%todo: 10, 7
\egroup
\end{document}