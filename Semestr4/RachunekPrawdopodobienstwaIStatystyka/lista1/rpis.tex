\documentclass[12pt]{article}
\usepackage{amsmath}
\usepackage[T1]{fontenc}
\usepackage{graphicx}
\usepackage{amsfonts}
\newcommand{\floor}[1]{\left\lfloor #1 \right\rfloor}	% podłoga
\newcommand{\ceil}[1]{\left\lceil #1 \right\rceil}		% sufit
\newcommand{\fractional}[1]{\left\{ #1 \right\}}		% część ułamkowa {x}
\newcommand{\abs}[1]{\left| #1 \right|}					% wartosc bezwzgledna / moc
\newcommand{\set}[1]{\left \{ #1 \right \}}				% zbiór elementów {a,b,c}
\title{Lista 1}
\author{Dominik Szczepaniak}
\begin{document}

\maketitle
\bgroup\obeylines

\section{Zadanie 1}
\begin{itemize}
    \item  $\sum_{k=0}^n \binom{n}{k} * p^k * (1-p)^{n-k} = 1$
    \item  $\sum_{k=0}^n k * \binom{n}{k} * p^k * (1-p)^{n-k} = np$
\end{itemize}
\section{Zadanie 2}

\section{Zadanie 3}

\section{Zadanie 4}

\section{Zadanie 5}

\section{Zadanie 6}

\section{Zadanie 7}

\section{Zadanie 8}

\section{Zadanie 9}

\section{Zadanie 10}

\egroup
\end{document}